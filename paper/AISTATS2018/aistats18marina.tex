\documentclass{article}

\usepackage{aistats2018}
 % If your paper is accepted, change the options for the package
% aistats2018 as follows:
%
%\usepackage[accepted]{aistats2018}
%
% This option will print headings for the title of your paper and
% headings for the authors names, plus a copyright note at the end of
% the first column of the first page.

\usepackage[utf8]{inputenc} % allow utf-8 input
\usepackage[T1]{fontenc}    % use 8-bit T1 fonts
\usepackage{hyperref}       % hyperlinks
\usepackage{url}            % simple URL typesetting
\usepackage{booktabs}       % professional-quality tables
\usepackage{amsfonts}       % blackboard math symbols
\usepackage{nicefrac}       % compact symbols for 1/2, etc.
\usepackage{microtype}      % microtypography
\usepackage[utf8]{inputenc} % allow utf-8 input
\usepackage[T1]{fontenc}    % use 8-bit T1 fonts
\usepackage{hyperref}       % hyperlinks
\usepackage{url}            % simple URL typesetting
\usepackage{booktabs}       % professional-quality tables
\usepackage{amsfonts,amsthm}       % blackboard math symbols
\usepackage{nicefrac}       % compact symbols for 1/2, etc.
\usepackage{microtype}      % microtypography
\usepackage{MnSymbol}
\usepackage{color}
\usepackage{graphicx} % more modern
\usepackage{subfigure} 
\usepackage{natbib}
\usepackage{algorithm}
\usepackage{algpseudocode}
\usepackage{lineno}
\usepackage{thmtools,thm-restate}
\usepackage{xr}
\usepackage{xcolor}
\usepackage{tikz}
\colorlet{BLUE}{blue}
\usepackage[normalem]{ulem}

%\linenumbers



\def\A{\mathcal{A}}
\def\ga{\gamma_\mathcal{A}}
\def\Co{C^{\circ}}
\def\Coa{C^{\circ}_{\!\!\mathcal{A}}}
\def\Ca{C_{\!\mathcal{A}}}
\def\go{\gamma^{\circ}}
\def\goa{\gamma^{\circ}_{\A}}
\newcommand{\dt}[2]{\langle#1,#2\rangle}
\def\Olasso{\Omega_{\rm Lasso}} % Lasso norm
\def\Ogl{\Omega_{\rm GL}} % group Lasso norm
\def\Olgl{\Omega_{\rm LGL}} % latent group Lasso norm
\def\Oks{\Omega_{\rm sp,k}} % k-support norm
\def\tr{{\rm tr}}
\newcommand\itgset[1]{[\!\![#1]\!\!]}
\def\RR{\mathbb{R}}
\def\EE{\mathbb{E}}
\newcommand{\tblue}[1]{\textcolor{blue}{#1}}
\newcommand{\tred}[1]{\textcolor{red}{#1}}
\newcommand\TODO[1]{\tblue{TODO: \texttt{#1}}}
%\newcommand\TODO[1]{}
\newcommand\GO[1]{\tred{GO: \texttt{#1}}}
%\newcommand\GOD[1]{\tred{GO: \texttt{\sout{#1}}}}
\newcommand\GOD[1]{}
\newcommand{\GOc}[1]{\todo[backgroundcolor=yellow!50, size=\scriptsize]{#1}}
\newcommand\ra{$\rightarrow$}

\newcommand{\repl}[2]{\textcolor{gray}{\sout{#1}}\: {\textcolor{red}{#2}}}
%\newcommand{\GOcD}[1]{\todo[backgroundcolor=yellow!50, size=\scriptsize]{\sout{#1}}}
\newcommand{\GOcD}[1]{}

\newcommand\OLD[1]{\tblue{#1}}
\newcommand\NEW[1]{\tred{#1}}
\def\st{\text{s.t.}}
\def\supp{\text{supp}}

\def\BIT{\begin{itemize}}
\def\EIT{\end{itemize}}
\def\BET{\begin{enumerate}}
\def\EET{\end{enumerate}}
\def\ra{$\:\rightarrow\:$}
\def\y{\mathbf{y}}
\def\x{\mathbf{x}}
\def\v{\mathbf{v}}
\def\u{\mathbf{u}}
\def\w{\mathbf{w}}
\def\s{\mathbf{s}}
\def\g{\mathbf{g}}
\def\h{\mathbf{h}}
\def\X{\mathbf{X}}
\def\RR{\mathbb{R}}
\def\ei{\varepsilon_i}
\def\ej{\varepsilon_j}
\def\T{\mathcal{T}}
\def\I{\mathcal{I}}
\def\D{\mathbb{R}^p}

\def\op{{\rm op}}
\def\supp{{\rm Supp}}
\def\st{\text{s.t.}}
\newcommand\normop[1]{|||#1|||_{\infty}}
\newcommand\normbop[1]{|\!|\!|#1|\!|\!|_{\infty,2}}
\newcommand\normopop[1]{|\!|\!|#1|\!|\!|_{\rm op,\rm op}}
\newcommand\ve[1]{\varepsilon^{{\scriptscriptstyle#1}}}
\newcommand\sss[1]{{\scriptscriptstyle#1}}
\def\uI{U^{\sss{I}}}
\def\xxi{\zeta}

%\newtheorem{thm}{Theorem}
%\newtheorem{prop}{Proposition}
%\newtheorem{fact}{Fact}
%\newtheorem{lemm}{Lemma}
%\newtheorem{coro}{Corollary}

\newtheorem{lemma}{Lemma}
\newtheorem{lemm}{Lemma}
\newtheorem{cor}{Corollary}
\newtheorem{prop}{Proposition}
\newtheorem{theorem}{Theorem}
\newtheorem{mydef}{Definition}
\newtheorem{ex}{Example}

%AISTATS 2017 experiments
\def \plasso {1000}
\def \nlasso {300}
\def \klasso {8}
\def \mlasso {10}
\def \lambdalasso {5}
\def \rholasso {5}
\def \sigmalasso {0.1}

\def \pwhs{50}
\def \pallwhs{1225}
\def \nwhs {1000}
\def \lambdawhs {5}
\def \sigmawhs {0.1}

\def \lambdacalhous {10^{-3}}

%graph



\begin{document}

% If your paper is accepted and the title of your paper is very long,
% the style will print as headings an error message. Use the following
% command to supply a shorter title of your paper so that it can be
% used as headings.
%
%\runningtitle{I use this title instead because the last one was very long}

% If your paper is accepted and the number of authors is large, the
% style will print as headings an error message. Use the following
% command to supply a shorter version of the authors names so that
% they can be used as headings (for example, use only the surnames)
%
%\runningauthor{Surname 1, Surname 2, Surname 3, ...., Surname n}

\twocolumn[

\aistatstitle{Learning the effect of latent variables in Gaussian Graphical models with unobserved variables}

\aistatsauthor{ Author 1 \And Author 2 \And  Author 3 }

\aistatsaddress{ Institution 1 \And  Institution 2 \And Institution 3 } ]


\begin{abstract}
Gaussian graphical models (GGM) have been widely used in many high-dimensional applications. Problems in which all relevant variables have been identified is extremely rare. The unobserved variables (or latent variables) can induce misleading dependence structure between observed variables. In this paper, we focus on a family of Latent Variable Gaussian graphical models (LVGGM), where the model is conditionally sparse given latent variables, but marginally non-sparse. In order to reveal latent variables and recover the structure of the complete model, we introduce a convex formulation with a new regularization imposing sparsity on latent factors.  We propose a tractable convex algorithm and study identifiability conditions. We show promising results on synthetic \TODO{and real datasets (remove)}.\\
\end{abstract}



\section{Introduction}
\label{intro}


Graphical models have emerged as useful tools for modelling complex systems. In many fields such as genomics and finance among others, we have to analyze high-dimensional data, where the dimension of the sample $p$ is of the same order or larger than the number of samples $n$, yielding to ill-conditioned problems. 
Regularization, structure, graphical lasso. 

In the context of Gaussian graphical models the central problem is often to estimate the inverse covariance matrix, also known as the concentration matrix. The sparsity pattern of the concentration matrix in such models correspond to the structure of the graph, i.e. the nonzeros of the concentration matrix correspond to the edges of the graphical model.. \\

LVGGM
Here model selection problem with latent variables. In applications where latent variables are missing, this may create non sparsity for some components on the graph, fully connecting all the observed variables connected to the latent variable, see Figure. 
Why learning GGM, then latent GGM to learn models with more interpretability and more efficient inference if sparse. With no a priori on the number of latent variables.

The resulting marginal precision matrix of the LVGGM has a sparse plus low-rank structure. In Chandrasekaran et al. (2012) they consider a regularized Maximum Likelihood approach, using $\ell_1$-norm for the sparse component and trace norm for the low-rank component. With the obtained decomposition one identifies the structure of the conditional graphical model on the observed variables and the subspace of latent variables. However the decomposition lacks to identify the effect of each latent variable separately since the subspace of latent variables does not have a unique decomposition. Choosing the orthogonal decomposition (SVD) gives us one choice.. In our setup we want to impose some additional structure to the low-rank component in order to be able to identify the complete LVGGM structure.  

In Richard et al they introduce a new norm that used as a regularizer yields to  estimates for low-rank matrices with multiple sparse factors that are provably more efficient statistically than the $\ell_1$ and trace norms.

In this work, we propose to impose more structure to the low rank matrix $L$ using a generalization of the trace norm introduced in Richard et al as a regularizer. This decomposition yields to a better interpretability of the Graphical model structure. We propose a convex formulation with a quadratic loss function that can be optimised efficiently throught the algorithm proposed in Vinyes et al.  Finally we study the identifiability of such decomposition. 

\subsection*{Notations}
$\itgset{p}$ denotes the set ${1,...,p}$. $|I|$ denotes the cardinality of a set $I$. If $M\in\RR^{n\times p}$ is a matrix, $I\itgset{n}$ and $J\itgset{p}$, $M_{I,J}\in\RR^{|I|\times |J|}$ is the submatrix obtained by selecting the rows indexed by $I$ and the columns indexed by $J$ in $M$.

%In many applications as genetics, finance...learning structure of graphical models. with more interpretability and more efficient inference if sparse. (cf drton for application examples). Speciffically, if the number of variables is very large compared to the number of samples
%We focus on learning undirected gaussian graphical models. A well studied problem is model selection where the inverse of the covariance, also called concentration matrix, is sparse. The nonzeros of the concentration matrix correspond to the edges of the graphical model. Graphical Lasso is the method of choice.
%Here model selection problem with latent variables. In applications where latent variables are missing, this may create non sparsity for some components on the graph, fully connecting all the observed variables connected to the latent variable, see Figure. 
%Why learning GGM, then latent GGM to learn models with more interpretability and more efficient inference if sparse. With no a priori on the number of latent variables.
%An interesting idea is to add sparsity to latent components. Theory on decomposition, more difficult if sparse latent components. 
%
%figure latent graphical model
 
\section{Related Work}
\label{related}
Two works, learning the structure on the observed variables and discovering latent variables and the structure of the complete latent gaussian graphical model
GGM l1\\
\citet{yuan2007model}] and \citet{banerjee2008model} proposed the graphical lasso estimator to impose sparsity on the concentration matrix.
A conditional independence graph is sometimes expected to have particular structure, such as ‘hub’ nodes with many neighbors. This motivated Defazio and Caetano (2012) to consider regularization with a sorted $\ell_1$ norm.
GGM with other regularizations\\
LGGM\\
\citet{chandrasekaran2010} propose a convex formulation to learn the structure on the observed variables and identifying the number of latent variables. \\
\citet{xu2017speeding}  In order to speed
up the estimation of the sparse plus low-rank components, we propose a sparsity constrained maximum likelihood estimator based on matrix factorization, and an efficient alternating gradient descent algorithm with hard thresholding to solve it.\\

\citet{hosseini2016learning}  present a framework to jointly learn both a network among observed variables and densely connected groups and overlaping groups of variables. revealing the existence of potential latent variables. Not convex, alternate.


ref in hosseini
\citet{marlin2009sparse} MAP estimation with a prior that prefers sparse graphs. In this paper, we use the stochastic block model as a prior
\citet{tan2015cluster} linkage on the variables

\citet{celik2014efficient}

Ambroise et al., 2009) uses an Expectation-
Maximization approach for variational estimation of the latent structure while inferring the network among the entire variable

\section{Problem setup}
\label{setup}

\subsection{Latent Variable GGM}
\label{sec:ggm}
We consider a multivariate Gaussian variable $(X_{O},X_{H})\in\RR^{p+h}$ where $O$ and $H$ are the set of indices of observed variables, $p=|O|$, and latent variables, $h=|H|$ respectively. We denote $\Sigma\in\RR^{(p+h)\times(p+h)}$ the complete covariance matrix and $K=\Sigma^{-1}$ the complete concentration matrix. Let $\hat{\Sigma}\in\RR^{(p+h)\times(p+h)}$ denote the empirical covariance matrix. We only have access to the empirical marginal covariance matrix $\hat{\Sigma}_{OO}$. It is well known that the marginal concentration matrix on the observed variables can be computed from the full concentration matrix as

\begin{align}
\Sigma_{OO}^{-1} = K_{OO}-K_{OH}K_{HH}^{-1}K_{HO}.
\end{align}

We assume that the original graphical model is sparse and that there is a small number of latent variables. 
%In GGMs, the sparsity pattern of the concentration matrix  directly corresponds to the graphical model structure. 
This implies that  $K_{OO}$ is a sparse matrix and that $K_{OH}K_{HH}^{-1}K_{HO}$ is a low-rank matrix, of rank at most $h$. Note that $\Sigma_{OO}^{-1}$ may not be sparse due to the addition of the term $K_{OH}K_{HH}^{-1}K_{HO}$.
%Since $K$ is a concentration matrix, it is positive semidefinite (p.s.d.), which results in  $K_{OO}$ and $K_{OH}K_{HH}^{-1}K_{HO}$ being p.s.d.. $\Sigma_{OO}^{-1}$ is also p.s.d. as it is also a concentration matrix. 
\citet{chandrasekaran2010} suggest to approximate $\hat{\Sigma}$  by $S-L$ where $S$ is sparse and $L$ is low rank, with $S-L$, $S$ and $L$ p.s.d. matrices in order to ensure the statistical interpretation of the different components. The authors propose propose a convex relaxation 
%Specifically, the nonzeros of the concentration matrix correspond to the edges of the graphical model. 
%The sparsity of the complete GGM implies that $K$ is sparse. Then, $K_{OO}$ should be a sparse matrix and assuming has a small number of latent variables is low rank. \citet{chandrasekaran2010} suggest to approximate $\hat{\Sigma}$  by $S-L$ where $S$ is sparse and $L$ is low rank, and propose a convex relaxation

\begin{align}
\label{opt_tr}
\min_{S,L} f(S-L)+\lambda\left(\gamma\|S\|_{1}+ \tr(L)\right) \quad s.t. \quad S-L \succeq 0 \quad L \succeq 0,
\end{align}

where the function $f$ is a loss function, $\lambda$ and $\gamma$ are the regularization parameters. The positivity constraint on $S$ has been dropped since it is implied by $S-L \succeq 0$ and $L \succeq 0$. Typically, in GGM seletion $f$ is the negative log-likelihood

\begin{align}
f_{ML}(M)&:=-\log\det(M) + \tr(M\hat{\Sigma}).
\end{align}

% LATER However $\textit{logdet}$ is not smooth\footnote{does not have Lipschitz gradients} preventing us from applying efficient algorithms from the family of conditional gradient, for which there is not convergence results for this setting.
 
Two other natural losses, that have the advantage of being quadratic, are the second order Taylor expansion arount the identity matrix of the log-likelihood $f_{T}$ and the score matching loss $f_{SM}$, introduced by \citet{hyvarinen2005estimation} and used for GGM estimation in \citet{lin2016estimation},
\begin{align}
f_{T}(M)&:=\frac{1}{2}\|\hat{\Sigma}^{1/2}M\hat{\Sigma}^{1/2}-I\|_2^2\\
f_{SM}(M)&:=\frac{1}{2}\tr(M^2 \hat{\Sigma})-\tr(M).
\end{align}

\citet{chandrasekaran2010} show that under appropriate technical conditions, the regularized maximum log-likelihood formulation (\ref{opt_tr}) provides estimates $(S_{n},L_{n})$ that have the same sparsity pattern and rank than $K_{OO}$ and $K_{OH}K_{HH}^{-1}K_{HO}$. 

%\TODO{explain better}
%The obtained low rank component $L_{n}$ retrieves the latent variable subspace. However it does not recover its structure, in particular the connectivity between the variables $(X_{O}$ and $X_{H})$ cannot be recoverd since ther is not a unique decomposition of the low rank component.\\
%
%
%The low rank component writes $UU^{\top}$, where $U\in\RR^{p\times h}$. If we assume that the latent variables are independent, i.e. $K_{HH}$ is  diagonal,  unicity of the decomposition $UU^{\top}$ can be achieved by imposing more structure on the low rank component, which we discuss in the next paragraph \ref{subsec:norm}. Under appropriate identifiability conditions, discussed in section \ref{sec:id}, it allows us to retrieve the structure of the full LVGGM. In order to obtain a convex formulation we introduce a norm for low rank \textit{positive semidefinite} (p.s.d.) matrices with multiple sparse factors. 


%\TODO{explain better: trying here}

The obtained low rank component $L_{n}$ retrieves the latent variable subspace. However, in general, estimates for $K_{HH}$ and  $K_{OH}$ cannot be obtained from estimate of $K_{OH}K_{HH}^{-1}K_{HO}$,         $L_{n}$. Therefore the connectivity between the latent variables and the connectivity between latent and observed variables cannot be recovered. 

%from the sparse plus low-rank decomposition.

%The low  rank component writes $UU^{\top}$, where $U\in\RR^{p\times h}$, and to recover the 

With the further assumption that the latent variables are independent,  i.e. $K_{HH}$ is  diagonal,  and writing the decomposition $S-UU^{\top}$, the structure of $U$, if unique,   gives the dependence structure  between the latent and the observed variables. In general unicity of $U$ is not guaranteed. If however we assume that the part of the graph $K_{OH}$ is also sparse, i.e. each latent variable is connected to a small number of observed variables, and provided that the sparsity patterns are not the same for each latent variable (for example sources connected to disjoint subsets of observed variables), we can ensure, under technical conditions discussed in section \ref{sec:id}, that the dependency structure between latent and observed variables, i.e the sparsity pattern of.$K_{OH}$, is recovered. If $U$ is unique, we recover the dependence structure  between the latent and the observed variables. Consequently we recover the structure of the full model.

%To impose sparsity in the low rank component..

%but can be achieved by imposing more structure on the low rank component, which we discuss in the next paragraph \ref{subsec:norm}. 

%With the further assumption that the latent variables are independent,  i.e. $K_{HH}$ is  diagonal,  unicity of the decomposition $UU^{\top}$ can be achieved by imposing more structure on the low rank component, which we discuss in the next paragraph \ref{subsec:norm}. Under appropriate identifiability conditions, discussed in section \ref{sec:id}, it allows us to retrieve the structure of the full LVGGM.

Unicity of $U$ can be achieved by imposing more structure on the low rank component. More precisely we want $U$ sparse with a small number of columns so $UU^{\top}$ is low rank. For this purpose we regularize with a matrix norm  introduced by \citet{richard2014tight}, which we review in the following
section.

%and that the decomposition is unique (see section \ref{sec:id} on identifiability).  Unicity of $K_{OH}$ is achieved imposing more structure on the low rank component, which we discuss in the next paragraph.
 
%talk here about structure of $K_{OO}-K_{OH}K_{HH}^{-1}K_{HO}.$  structure $S-L$ explain\\

%and three losses ML and the two quadratic\\

%$UU$ decomposition with sparsity on columns... Transition, in order to obtain convex formulation..\\
%
%\begin{align}
%FIGURE ?
%\end{align}

%\TODO{transition}

\subsection{Positive-rank($k$) and its relaxation}
\label{subsec:norm}

\citet{richard2014tight} propose a new matrix norm that yields estimates for low-rank matrices with multiple sparse factors. In particular, these authors define a norm for p.s.d. matrices \footnote{In fact it is a gauge. For more on gauges see \citet{chandrasekaran2010convex} and references therein.}  that yields to estimates with sparse and p.s.d factors. In this section we review the concepts introduced by \citet{richard2014tight} and introduce the positive-rank($k$). We assume that the sparsity of the factors is known and fixed and discuss a generalization for factors of different sparsity levels at the end of the section.

The following definition generalizes the notion of rank for p.s.d matrices,
\begin{mydef}
(positive-rank($k$)) For a p.s.d  matrix $Z\in\RR^{p\times p}$ and for $k>1$ we define its positive-rank($k$) as the optimal
value of the optimization problem:
\begin{align}
\min \|c\|_0 \quad \text{s.t.} \quad Z=\sum_{i} c_i u_i u_{i}^\top, \quad c_i\in \RR^{+} \quad u_{i}\in\RR^p  :   \|u_{i}\|_0 \leq k, \|u_{i}\|_2 = 1.
\end{align}
\end{mydef}
Note that not all p.s.d. matrices can have such a decomposition, so the positive-rank($k$) can be infinite. This is in particular the case for low-rank non sparse matrices like $11^{\top}$. 

We can derive a convex relaxation of the positive-rank($k$), the convex function $\Omega_{pos,k}$, defined as follows :
%the atomic norm%\footnote{$\Omega_{k,\succeq}$ is not a norm but only a gauge because the set $\{u u^\top \in\RR^p  :   \|u\|_0 \leq k, \|u\|_2 = 1\}$  is not centrally symmetric} $\Omega_{k,\succeq}$. The next lemma provides an explicit formulation of the dual norm
\begin{mydef}
($\Omega_{pos,k}$) For a p.s.d  matrix $Z\in\RR^{p\times p}$ 
\begin{align}
\Omega_{pos,k}(Z):=\min \|c\|_1 \quad s.t. \quad Z=\sum_{i} c_i u_i u_{i}^\top, \quad c_i\in \RR^{+} \quad u_{i}\in\RR^p  :   \|u_{i}\|_0 \leq k, \|u_{i}\|_2 = 1
\end{align}
Equivalently, as shown in the lemma 7 of \citet{richard2014tight},
\begin{align}
\Omega_{pos,k}(Z):=\inf_{Z^I, I\in\mathcal{G}^p_k} \sum_{I}\tr(Z^I) \quad \text{s.t.} \quad Z^I\succeq 0 ,\quad\supp(Z^I)\subset I\times I
\end{align}
%where  $\mathcal{A}_{I,\succeq}:=\{Z_I\succeq 0 ,\supp(Z_I)=I\times I\}$.
\end{mydef}

We can have $\Omega_{pos,k}(Z)=+\infty$ even if $Z$ is p.s.d., if $Z$ cannot be decomposed in $k$-sparse,  rank-1 p.s.d. factors, as it is the case for $11^{\top}$. In the next lemma we give the polar norm of $\Omega_{pos,k}$ :

\begin{lemm}
\label{lem:LMO}
Let $Y\in\RR^{p\times p}$ be a symmetric matrix. The polar norm of $\Omega_{pos,k}$ writes
\begin{align}
{\Omega_{pos,k}^{\circ}}(Y)= \max_{I\in\mathcal{G}^p_k}\lambda^{+}_{max}(Y_{II}).
\end{align}
\end{lemm}

It is important to note that polar norm $\Omega_{pos,k}^{\circ}$ is NP-hard to compute. Indeed it consits of a generalization to general symmetric matrices of the rank-one sparse PCA problem for p.s.d matrices $XX^{\top}$,
\begin{align*}
\min u^{\top}XX^{\top}u \quad s.t.  \|u\|_0 \leq k,\quad \|u\|_2 = 1
\end{align*}
which is known to be an NP-hard problem \citep{moghaddam2008sparse}.

%In fact $\Omega_{k,\succeq}$ is an atomic gauge\footnote{$\Omega_{k,\succeq}$ is not a norm but only a gauge because the set $\{u u^\top \in\RR^p  :   \|u\|_0 \leq k, \|u\|_2 = 1\}$  is not centrally symmetric} associated to the atomic set $\{uu^{\top}  : \|u_{i}\|_0 \leq k, \|u_{i}\|_2 = 1\}$ and the LMO in lemma \ref{lem:LMO} is the polar gauge $\Omega^{\circ}_{k,\succeq}$.

\TODO{explain a lot more}

$\Omega_{pos,k}$ can be generalised for different sparsity levels as
\begin{align}
\Omega_{\succeq}(Z):=\inf \sum_{k,i}w_{k}c_i^k \quad \text{s.t.} \quad Z=\sum_{k,i} c_i^k u_i^k u_{i}^{k\top}, \quad c_i^k\in \RR^{+} \quad u_{i}^k\in\RR^p  :   \|u_{i}^k\|_0 \leq k, \|u_{i}^k\|_2 = 1,
\end{align}
where $w_{k}$ is an increasing cardinality function that penalizes each sparsity level $k$ by $w_{k}$. We illustrate this generalization in the experiments.\\

\TODO{better explained here}
$\Omega_{pos,k}$ can be generalised for different sparsity levels.

In the vector case \TODO{the norm on non symetric psd... when to explain ?}  boils down to the $k$-support norm \citep{argyriou2012sparse} and different sparsity levels are considered in the ordered weighted $\ell_1$ (OWL) norm introduced by \citet{bogdan2013statistical}. We can naturally extend this idea to the matrix counterpart $\Omega_{pos,w}$ of this norms 

\begin{align}
\Omega_{pos,w}(Z):=\inf \sum_{k,i}w_{k}c_i^k \quad \text{s.t.} \quad Z=\sum_{k,i} c_i^k u_i^k u_{i}^{k\top}, \quad c_i^k\in \RR^{+} \quad u_{i}^k\in\RR^p  :   \|u_{i}^k\|_0 \leq k, \|u_{i}^k\|_2 = 1,
\end{align}
where $w_{k}$ is an increasing cardinality function that penalizes each sparsity level $k$ by $w_{k}$. We illustrate this generalization in the experiments.


%1) normes liees a la k-support norm\\
%2) autres tailles considerees dans le cas vectoriel OWL \citet{bogdan2013statistical}
%3) et que on peut naturellement ... the matrix counterpart of this norms. A la condition que les poids soient decroissants.



\subsection{Convex formulation}
%Let $(x_1,..,x_n)$ be $n$ samples of dimension $p$ and   $\hat{\Sigma}$ the empirical covariance. A natural way to approximate a given sample covariance matrix by a model which the concentration matrix decomposes into a sparse and $k$-low-rank matrix is a regularized using $\ell_1$ norm for recovering sparse component and the convex relaxation of $k$-rank for recovering the $k$-low-rank component. Hence we consider the following convex optimization problem
%\begin{align}
%\label{opt}
%\min_{S,L} f(S-L)+\mu\|S\|_{1}+\lambda\Omega_k(L) \quad s.t. \quad S-L \succeq 0 \quad L \succeq 0,
%\end{align}
%where  $f$ represents the loss function.
% Typically,in Graphical model seletion $f$ is the log-likelihood, as proposed in \citet{chandrasekaran2010}. However \textit{logdet} is not Lipschitz preventing us from applying efficient algorithms of the family of conditional gradient, for which there is not convergence results for this setting.  Two other natural losses, which have the advantage of being quadratic, are the second order taylor expansion of the log-likelihood $f_{T}$ and the score matching loss $f_{SM}$, introduced by \citet{hyvarinen2005estimation} and used for graphical model estimation in \citet{lin2016estimation}.
%\begin{align}
%f_{SM}(K)&:=\frac{1}{2}\tr(K^2 \hat{\Sigma})-\tr(K) \\
%f_{T}(K)&:=\frac{1}{2}\|\hat{\Sigma}^{1/2}K\hat{\Sigma}^{1/2}-I\|_2^2.
%\end{align}

We use $\Omega_{k,\succeq}$ to impose structure on the low rank component and consider the following convex optimization problem,
\begin{align}
\label{opt}
\min_{S,L} f(S-L)+\lambda\big(\gamma\|S\|_{1}+\Omega_{k,\succeq}(L)\big) \quad s.t. \quad S-L \succeq 0 \quad L \succeq 0.
\end{align}
Since the norm $\Omega_k$ only provides symmetric p.s.d matrices, as a sum of p.s.d rank-one matrices, the nonnegativity constraint on $L$ can be dropped. The regularization $\gamma\|S\|_{1}+\Omega_k(L)$ defines an atomic norm on matrices as shown in the following lemma

\TODO{plus une remarque q'un lemme}
\begin{lemm} $\bar{\Omega}(M):=\inf\{\gamma\|A\|_{1}+\Omega_k(B)\mid M=A+B\}$ is an atomic norm and its polar norm is given by 
\begin{align*}
{\bar{\Omega}^{\circ}}(Y)=\max\left(\frac{\|Y\|_{\infty}}{\gamma},\Omega_{k,\succeq}^{\circ}(Y)\right).
\end{align*}
\end{lemm}
In order to rewrite our problem as a simple convex regularized by $\bar{\Omega}$, we drop the nonegativity constraint on $S-L$. Thus our problem is rewritten as
\begin{align}
\label{opt_at}
\min_{M} f(M)+ \bar{\Omega}(M) \quad s.t. \quad M \succeq 0,
\end{align}
and $M$ writes as a sum of atoms of $\ell_1$ and atoms of $\Omega_k$. Therefore we can recover a sparse component and a low-rank component with multiple sparse factors.\\

\subsection{Algorithm}
We start by reviewing Frank Wolfe (FW) algorithm \citep{frank1956algorithm,LacosteFCFW}. 
FW algorithm, also known as conditional gradient, is particularly well suited for constrained quadratic optimization of the form
\begin{align*}
\min f(x) \quad s.t. \quad x\in \mathcal{C}
\end{align*}
where $\mathcal{C}$ is convex and bounded. In particular $\mathcal{C}$ can be the convex hull of a set of atoms $\A$  In each iteration FW needs to solve a linear minimization oracle (LMO) which solves the optimization problem
\begin{align}
\rm{{LMO}}_{\mathcal{C}}(y) := \arg\min_{z \in \mathcal{C}} \left\langle y,z \right\rangle.
\end{align}
At each iteration FW selects a new atom $a^t$ from $\mathcal{C}$ querying the LMO and computes the new iterate as a convex combination of $a^t$ and the old iterate $x^t$. The convex update can be done by line search. Another variant, called the fully corrective FW (FCFW), is discussed in \citet{LacosteFCFW} consits of finding the convex combination of all previously selected atoms $(a^i)_{i<t}$. 
 Generalization for regularized problems. We choose to consider quadratic losses $f_{T}$ and $f_{SM}$, introduced in section \ref{sec:ggm}, so that we can apply an efficient algorithm recently proposed by \citet{vinyes2017}(\texttt{FCG}). The algorithm consists of applying a Fully Corrective Frank Wolfe \citep{LacosteFCFW}, for which we need to compute the LMO at each iteration. 


We propose to use a Truncated Power Iteration (TPI) heuristic introduced by \citet{yuan2013truncated} to approximate the oracle. In order to apply the TPI to general symmetric matrices, we add the frobenius norm of the matrix before applying it. \\


In practice, and in order to reduce the number of calls to TPI we propose to solve problem (\ref{opt_at}) using a working set algorithm which solves a sequence
of problems of the form 
\begin{align}
\label{opt_ps}
\min_{K} f(K)+ \bar{\Omega}_{\mathcal{A}^{t}}(K) \quad s.t. \quad K \succeq 0,
\end{align}
where $\bar{\Omega}_{\mathcal{A}^{t}}$ is the atomic norm on a growing sequence of atomic sets $\mathcal{A}^{1}\subset \mathcal{A}^{2} \subset ... \subset \mathcal{A}^{t}$. We start with the atomic set of $\ell_1$-norm, $\mathcal{A}^{1}=\mathcal{A}_{\ell_{1}}$\footnote{$\mathcal{A}_{\ell_{1}}=\{\pm e_i e_j^{\top}\}$ where $e_i$ is the canonical basis of $\RR^p$}, and conider the sequence $\mathcal{A}^{t}=\mathcal{A}_{\ell_{1}}\cup \big\{uu^{\top} \quad \text{s.t.} \quad \supp(u)\subset \mathcal{S}, \|u\|_{2}=1 \big\}$ for a growing sequence of working sets $\mathcal{S}$. We use algorithm of \texttt{FCG} to solve each problem (\ref{opt_ps}). The method allows us to recover all the active atoms and its weights. in particular we recover the sparse component $S$ and the low-rank sparse component $UU^{\top}$.  In order to build the  sequence of working sets $\mathcal{S}$ we use TPI for p.s.d matrices.  The procedure is explained in algorithm \ref{alg:colgen_ggm} and consists on three loops: first we augment the atomic set $\mathcal{A}^{t}$ for atoms of  $\Omega_{k,\succeq}$, then we solve the subproblem (\ref{opt_ps}) with \texttt{FCG} that internally uses an active-set procedure.

%explain three loops, first we add support, then atoms on support, then as 

\begin{algorithm}
\caption{Column generation}
\label{alg:colgen_ggm}
\begin{algorithmic}[1]
\State\textbf{Require: } $f$ convex differentiable, tolerance $\epsilon$ 
\State\textbf{Initialization: } $K^{1}=0$,  $S^{1}=0$, $U^{1}=0$, $\mathcal{A}_{\ell_{1}}=\{\pm e_i e_j^{\top}\}$, $\mathcal{A}^{1}_{\Omega}=\varnothing$, $t=1$
\While{c=\texttt{true}}
\State Compute $K^{t},S^{t},U^{t}$ applying \texttt{FCG} on problem (\ref{opt_ps}) restricted to atomic set $\mathcal{A}_{\ell_{1}} \cup \mathcal{A}_{\Omega}^{t}$ with warm start solution
\State $G^{t}\gets -\nabla(K^t)$
\State $I \gets \texttt{TPI}(G^{t})$
\If { $\lambda^{+}_{max}(G^{t}_{II})>\lambda(1+\epsilon)$}
\State $\mathcal{A}_{\Omega}^{t+1}\gets \mathcal{A}_{\Omega}^{t}
 \cup \big\{uu^{\top} \quad \text{s.t.} \quad \supp(u)=I, \|u\|_{2}=1 \big\} $
\Else { $c=\texttt{false}$}
\EndIf
\State $t \gets t+1$
\EndWhile
\end{algorithmic}
  \end{algorithm}


% \begin{algorithm}
%   \caption{Column generation}
%\label{alg:colgen}
%\begin{algorithmic}[1]
%\State\textbf{Require: } $f$ convex differentiable, tolerance $\epsilon$ 
%\State\textbf{Initialization: } $x^0=0$, $A^0=\varnothing$, $k_0=0$, $t=1$
%\Repeat
%\State $a_{t}\gets \arg\max_{a \in \A} \dt{-\nabla f(x^{t-1})}{a}$ %\Comment New atom selection : using oracle for $\A$
%\State $A^t \gets [A^{t-1},a_{t}]$
%\State $c^t\gets \arg\min_{c \geq 0} f \big (A^{t} c \big ) + \|c\|_1$ %\Comment Relaxed primal : solved with active-set procedure
%\State $I\gets \{i \mid c_i^t>0\}$,
%%\State $k_t \gets |I|$ 
%\State $c^t \gets c^t_{I}$
%\State $A^t \gets A^t_{\cdot,I}$
%\State $S^t\gets A^{t}c^t$
%\State $L^t\gets A^{t}c^t$
%\State $t \gets t+1$
%\Until $\max_{a \in \A} \dt{-\nabla f(x^{t-1})}{a}\leq \epsilon$
%\end{algorithmic}
%  \end{algorithm}







\section{Identifiability}
\label{sec:id}
It is beyond the scope of thispaper to provide in this a proof comparable to that of \citet{chandrasekaran2010} which would show that for optimisation problem (\ref{opt_at}) the support of $S^*$ and the correct rank for each block is recovered. However, we prove a first result towards this goal namely that if  $M^{\ast}$ is a matrix that writes as  $M^{\ast}= S^{\ast}-L^{\ast}$, with $S^{\ast}$ an unknown sparse matrix and an unknown  low $k,\succeq$-rank matrix, that writes  $L^{\ast}=\sum_{I\in\mathcal{I}L^{I}}$. We want to know, given $M^{\ast}$, ther exist a unique decomposition $\left(S,(L^{I})_{I\in\mathcal{I}}\right)$ satisfying this conditions. We consider the case where the sets $I\in\mathcal{I}$ do not overlap.  \\

More formally we want to know under which conditions the following optimization problem 
\begin{align}
\min \gamma\|S\|_1+\Omega_{k,\succeq}(L) \quad s.t. \quad M=S+L
\end{align}
has a unique solution.\\

In order to recover such a decompostion  $\left(S,(L^{I})_{I\in\mathcal{I}}\right)$, identifiability conditions are necessary, that is there is no ambiguity between the subspaces containing the different components.  \citet{chandrasekaran2011rank} introduce natural conditions formulated in terms of incoherence between tangent subspaces. We define the tangent spaces for sparse matrices and for our low-rank sparse components.

For any matrix $A\in\RR^{p\times p}$, the tangent space $\T_s(A)$ at $A$ with respect to the variety of matrices with size of support less than the size of the support of $A$ is given by
$$
\T_s(A)=\{M \in \RR^{p \times p}\mid  \: M=M^\top, \: \supp(M)\subset \supp(A)\}
$$

For any p.s.d. matrix $UU^{\top}\in\RR^{p\times p}$  the tangent space $\T(U)$ at $UU^{top}$ with respect to the variety of matrices of fixed same rank given by
$$
\T(U):=\{M \in \: M=UX^\top+XU^\top,\: X \in \RR^{\text{rank}(U) \times p} \}.
$$
In our setting we need to consider the tangent subspace for a matrix $UU^{\top}\in\RR^{p\times p}$ with fixed support $I$. This subspace is
We consider several types of subspaces
$$\T_I(U):=\{M \in \bar{\T}_I\mid \: M=UX^\top+XU^\top,\: X \in \RR^{\text{rank}(U) \times p} \},$$
where 
$$
\bar{\T}_I:=\{M \in \RR^{p \times p}\mid  M=M^\top, \: \supp(M)\subset I \times I\}.
$$

Let $\T^c_I(U)$ denote the orthogonal complement\footnote{Note in particular that it is not the orthogonal complement in the entire space.} of $\T_I(U)$ in $\bar{\T}_I$. The projections on the corresponding subspaces are $\mathcal{P}_{\T_s(A)}(M)=M_S$ and $\mathcal{P}_{\T_I(U)}(M)=\mathcal{P}_{U}(M_{II})$ with $$\mathcal{P}_U(M):=M-(I-UU^\top)M(I-UU^\top).$$ 

%possible we introduce a notion of incoherence between the subspaces containing the different components. Following ideas from , we extend identifiability conditions for our setting and show that  our assumptions yields results that are different than the one obtained in \citet{chandrasekaran2011rank}. 
The following quantities define incoherence between subspaces
\begin{eqnarray*}
\xxi(\T_s(S),\T_J(V))&=&\max \{ {\|M_S\|_\infty} \mid M \in \T_J(V), \: \|M\|_{\op}\leq 1\}\\
\xxi(\T_I(U),\T_s(S))&=&\max \{ {\|\mathcal{P}_{U}(M_{II})\|_{\op}} \mid M \in \T_s(S), \: \|M\|_{\infty}\leq 1\}\\
\xxi(\T_I(U),\T_J(V))&=&\max \{ {\|\mathcal{P}_{U}(M_{II})\|_{\op}} \mid  M \in \T_J(V), \: \|M\|_{\op}\leq 1\}\\
\xi(\T_J(V))&=&\max \{ {\|M\|_\infty} \mid M \in \T_J(V), \: \|M\|_{\op}\leq 1\}\\
\mu_I(\T_s(S))&=&\max \{ {\|M_{II}\|_\op} \mid M \in \T_s(S), \: \|M\|_{\infty}\leq 1\}\\
\end{eqnarray*}

\begin{thm}
\label{theo:two}
If $M^*=S^*-L^*$ with $L^*=\sum_{I \in \mathcal{I}} L^{\sss{I}}$, $\supp(L^{\sss{I}}) \subset I \times I$ for $\I \subset \mathcal{G}^p_k$ %(where $\mathcal{G}^p_k$ is the collection of all subsets of size $k$ of $[\![p]\!]$)
and $L^{\sss{I}}=U^{\sss{I}} D^{\sss{I}} {U^{\sss{I}}}^\top$ the eigenvalue decomposition of $L^{\sss{I}}$.
Let $\mu'_I=\xxi(\T_I(U),\T_s(S))$, $\xi'_I :=\xxi(\T_s(S),\T_I(U))$, $\mu_I:=\mu_I(\T_s(S))$, $\xi_I:=\xi(\T_I(U))$,
$$\alpha_I:=\frac{1-\mu'_I\xi'_I}{1+\max(\gamma \mu'_I,\frac{1}{\gamma} \xi'_I)} \qquad \text{and} \qquad \epsilon_I :=\max \big (\|\mathcal{P}_{\T_I(\uI)}(\gamma \text{sign}(S^*))\|_{\op}, \gamma^{-1}\|\uI {\uI}^\top\|_{\infty} \big ).$$

Assume that for all $(I,I') \in \I \times \I, I \cap I'=\varnothing$. If $$\displaystyle\max_{I \in \I} \max(\mu_I,\xi_I) (1+\frac{\epsilon_I}{\alpha_I})<1, \quad \displaystyle \max_{J \in \mathcal{G}^p_k} \lambda_{\max}^+\big (S^*_{\bar{I}^c \cap (J \times J)} \big ) <1$$ %\quad 
%\text{and}  \quad \max_{J \in \mathcal{G}^p_k, J \neq I}\lambda_{\max}^+(\uI_J{\uI_J}^\top)<1,$$
with $\displaystyle \bar{I}=\bigcup_{I \in \I} I \times I$,
then the solution of the optimization problem $$\displaystyle \min \gamma \|S\|_1+\Omega_{k,\succeq}(L) \quad \text{\st} \quad M=S+L$$
 the decomposition given by $\big (S^*,(D^{\sss{I}} \big)_{I \in \I},(\uI)_{I \in \I} \big ) as a solution.$ 
\end{thm}

The proof can be found in the appendix. We follow the general proof scheme of \citet{chandrasekaran2011rank}. The optimization problem admits the unique announced solution if it satisfies first order gradient condition.


\section{Experiments}
\label{experiments}

\subsection{Simulated data}
k-3-blocs+tree structure\\

different size blocs\\

overlapping blocs\\

\subsection{dataset}

\subsection*{Conclusion}
We consider a family of latent variable Gaussian graphical model whose concentration matrix has a sparse plus low-rank structure. We introduce a regularization to impose more structure on the low rank component, as a sum of multiple low-rank and $k$-sparse factors, and derive a convex formulation. We propose an efficient algorithm to optimize this problem and show promising results for recovering the structure of the complete GGM in the experiments. We provide a first result for identifiability condition. Futurework includes extending the framework to multiple levels of sparsity and generalizing identifiability conditions for recovery guarantees on our optimisation problem .
%
%\subsubsection*{Acknowledgements}
%
%Use unnumbered third level headings for the acknowledgements.  All
%acknowledgements go at the end of the paper.

%\subsubsection*{References}
\bibliographystyle{apalike}
\bibliography{lvggm}

\end{document}
