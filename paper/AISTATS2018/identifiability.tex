\section{Identifiability}
\label{sec:id}
%It is beyond the scope of this paper to provide in this a proof comparable to that of \citet{chandrasekaran2010} which would show that for . 

In this section, we prove a first result towards showing that the optimisation problem (\ref{opt_at}) recovers the support of $S^*$ and the correct rank for each block under technical assumptions. We however do not show exactly that result, which would be beyond the scope of this work. 

Following the work of~\citet{chandrasekaran2011rank}, we show that if  $M^{\ast}$ is a matrix that writes as  $M^{\ast}= S^{\ast}-L^{\ast}$, with $S^{\ast}$ an unknown sparse matrix and an unknown  low $k,\succeq$-rank matrix, which is itself a sum $L^{\ast}=\sum_{I\in\mathcal{I}L^{I}}$ of matrices $L^{I}$ each supported on $I \times I$ and with decomposition . We want to know, given $M^{\ast}$, ther exist a unique decomposition $\left(S,(L^{I})_{I\in\mathcal{I}}\right)$ satisfying this conditions. We consider the case where the sets $I\in\mathcal{I}$ do not overlap.  \\

More formally we want to know under which conditions the following optimization problem 
\begin{align}
\min \gamma\|S\|_1+\Omega_{k,\succeq}(L) \quad s.t. \quad M=S+L
\end{align}
has a unique solution.\\

In order to recover such a decompostion  $\left(S,(L^{I})_{I\in\mathcal{I}}\right)$, identifiability conditions are necessary, that is there is no ambiguity between the subspaces containing the different components.  \citet{chandrasekaran2011rank} introduce natural conditions formulated in terms of incoherence between tangent subspaces. We define the tangent spaces for sparse matrices and for our low-rank sparse components.

For any matrix $A\in\RR^{p\times p}$, the tangent space $\T_s(A)$ at $A$ with respect to the variety of matrices with size of support less than the size of the support of $A$ is given by
$$
\T_s(A)=\{M \in \RR^{p \times p}\mid  \: M=M^\top, \: \supp(M)\subset \supp(A)\}
$$

For any p.s.d. matrix $UU^{\top}\in\RR^{p\times p}$  the tangent space $\T(U)$ at $UU^{top}$ with respect to the variety of matrices of fixed same rank given by
$$
\T(U):=\{M \in \: M=UX^\top+XU^\top,\: X \in \RR^{\text{rank}(U) \times p} \}.
$$
In our setting we need to consider the tangent subspace for a matrix $UU^{\top}\in\RR^{p\times p}$ with fixed support $I$.
We consider several subspaces
$$\T_I(U):=\{M \in \bar{\T}_I\mid \: M=UX^\top+XU^\top,\: X \in \RR^{\text{rank}(U) \times p} \},$$
where 
$$
\bar{\T}_I:=\{M \in \RR^{p \times p}\mid  M=M^\top, \: \supp(M)\subset I \times I\}.
$$

Let $\T^c_I(U)$ denote the orthogonal complement\footnote{Note in particular that it is not the orthogonal complement in the entire space.} of $\T_I(U)$ in $\bar{\T}_I$. The projections on the corresponding subspaces are $\mathcal{P}_{\T_s(A)}(M)=M_S$ and $\mathcal{P}_{\T_I(U)}(M)=\mathcal{P}_{U}(M_{II})$ with $$\mathcal{P}_U(M):=M-(I-UU^\top)M(I-UU^\top).$$ 

%possible we introduce a notion of incoherence between the subspaces containing the different components. Following ideas from , we extend identifiability conditions for our setting and show that  our assumptions yields results that are different than the one obtained in \citet{chandrasekaran2011rank}. 
The following quantities define incoherence between subspaces
\begin{eqnarray*}
\xxi(\T_s(S),\T_J(V))&=&\max \{ {\|M_S\|_\infty} \mid M \in \T_J(V), \: \|M\|_{\op}\leq 1\}\\
\xxi(\T_I(U),\T_s(S))&=&\max \{ {\|\mathcal{P}_{U}(M_{II})\|_{\op}} \mid M \in \T_s(S), \: \|M\|_{\infty}\leq 1\}\\
\xxi(\T_I(U),\T_J(V))&=&\max \{ {\|\mathcal{P}_{U}(M_{II})\|_{\op}} \mid  M \in \T_J(V), \: \|M\|_{\op}\leq 1\}\\
\xi(\T_J(V))&=&\max \{ {\|M\|_\infty} \mid M \in \T_J(V), \: \|M\|_{\op}\leq 1\}\\
\mu_I(\T_s(S))&=&\max \{ {\|M_{II}\|_\op} \mid M \in \T_s(S), \: \|M\|_{\infty}\leq 1\}\\
\end{eqnarray*}

\begin{thm}
\label{theo:two}
If $M^*=S^*-L^*$ with $$L^*=\sum_{I \in \mathcal{I}} L^{\sss{I}},  \quad \supp(L^{\sss{I}}) \subset I \times I$$ for $\I \subset \mathcal{G}^p_k$ %(where $\mathcal{G}^p_k$ is the collection of all subsets of size $k$ of $[\![p]\!]$)
and $L^{\sss{I}}=U^{\sss{I}} D^{\sss{I}} {U^{\sss{I}}}^\top$ the eigenvalue decomposition of $L^{\sss{I}}$.

Let $$\mu'_I=\xxi(\T_I(U),\T_s(S)), \quad  \xi'_I :=\xxi(\T_s(S),\T_I(U)), \quad \mu_I:=\mu_I(\T_s(S)) , \quad \xi_I:=\xi(\T_I(U)),$$
and 
$$\alpha_I:=\frac{1-\mu'_I\xi'_I}{1+\max(\gamma \mu'_I,\frac{1}{\gamma} \xi'_I)} \qquad \text{and} \qquad \epsilon_I :=\max \big (\|\mathcal{P}_{\T_I(\uI)}(\gamma \text{sign}(S^*))\|_{\op}, \gamma^{-1}\|\uI {\uI}^\top\|_{\infty} \big ).$$

Assume that for all $(I,I') \in \I \times \I, I \cap I'=\varnothing$. If $$\displaystyle\max_{I \in \I} \max(\mu_I,\xi_I) (1+\frac{\epsilon_I}{\alpha_I})<1, \quad \displaystyle \max_{J \in \mathcal{G}^p_k} \lambda_{\max}^+\big (S^*_{\bar{I}^c \cap (J \times J)} \big ) <1$$ %\quad 
%\text{and}  \quad \max_{J \in \mathcal{G}^p_k, J \neq I}\lambda_{\max}^+(\uI_J{\uI_J}^\top)<1,$$
with $\displaystyle \bar{I}=\bigcup_{I \in \I} I \times I$,
then the  decomposition given by $\big (S^*,(D^{\sss{I}} \big)_{I \in \I},(\uI)_{I \in \I} \big )$ is solution to the optimization problem $$\displaystyle \min \gamma \|S\|_1+\Omega_{k,\succeq}(L) \quad \text{\st} \quad M=S+L.$$
\end{thm}

The proof can be found in the appendix. We follow the general proof scheme of \citet{chandrasekaran2011rank}. The optimization problem admits the unique announced solution if it satisfies first order gradient condition.
