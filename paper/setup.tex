\section{Problem setup}
\label{setup}

\subsection{Latent Variable GGM}
We consider the problem of Latent-Variable Gaussian Graphical Model (LVGGM) selection. Let $O$ and $H$ be the set of indexes of observed variables and latent variables respectively. Let $p$ be the number of observed variables, $|O|=p$, and  $h$ the number of latent variables $|H|=h$. We denote $\Sigma\in\RR^{(p+h)\times(p+h)}$ the complete covariance matrix and $K=\Sigma^{-1}$ the complete concentration matrix. $\hat{\Sigma}\in\RR^{(p+h)\times(p+h)}$ denotes the empirical covariance matrix. We only have access to the empirical marginal covariance matrix $\hat{\Sigma}_{OO}$. The Schur complement of $K$ with respect to block $K_{HH}$ gives 

\begin{align}
\Sigma_{OO}^{-1} = K_{OO}-K_{OH}K_{HH}^{-1}K_{HO}.
\end{align}

$K_{OO}$ specifies conditional dependency of observed variables given the latent variables. If we assume that the latent variables are independent, i.e. $K_{HH}$ diagonal, and that the decomposition is unique (see section on identifiability), $K_{OH}$ specifies the effects of latent variables on observed variables. Unicity of $K_{OH}$ is achieved imposing more structure on the sparse component, which we discuss in the next paragraph. Under appropriate identifiability conditions (see section) it allows us to retrieve the structure of the full LVGGM. Note that Sparse+Low rank decompositions only retrieve latent variable subspace but not its structure.

talk here about structure of $K_{OO}-K_{OH}K_{HH}^{-1}K_{HO}.$  structure $S-L$ explain\\

and three losses ML and the two quadratic\\

$UU$ decomposition with sparsity on columns... Transition, in order to obtain convex formulation..\\

\begin{align}
FIGURE ?
\end{align}

\subsection{$k$-rank and its relaxation}
\label{subsec:norm}
%$(k, q)$-rank of a matrix as the minimum number of left and right factors, having respectively $k$ and $q$ nonzeros, required to reconstruct a matrix. 

Definition k-rank for positive semidefinite (p.s.d.) matrices
\begin{mydef}
($k,\succeq$-rank) For a p.s.d  matrix $Z\in\RR^{p\times p}$ \TODO{assuming that the decomposition is possible} we define its $k,\succeq$-rank as the optimal
value of the optimization problem:
\begin{align}
\min \|c\|_0 \quad \text{s.t.} \quad Z=\sum_{i} c_i Z_i, \quad (c_i,Z_i)\in \RR^{+}\times\mathcal{A}_{k,\succeq}
\end{align}
where  $\mathcal{A}_{k,\succeq}:=\{u u^\top \in\RR^p  :   \|u\|_0 \leq k, \|u\|_2 = 1\}$ , that is $\mathcal{A}_k$ is the set of $p$-dimensional unit vectors with at most $k$ non-zero components.
\end{mydef}

The convex relaxation of the $k,\succeq$-rank, is the gauge $\Omega_{k,\succeq}$. 

\begin{mydef}
($\Omega_{k,\succeq}$) For a p.s.d  matrix $Z\in\RR^{p\times p}$ \TODO{assuming that the decomposition is possible} :
\begin{align}
\Omega_{k,\succeq}(Z):=\{\min \|c\|_1 \quad \text{s.t.} \quad Z=\sum_{i} c_i Z_i, \quad (c_i,Z_i)\in \RR^{+}\times\mathcal{A}_{k,\succeq}\}
\end{align}
Equivalently, 
\begin{align}
\Omega_{k,\succeq}(Z):=\{\min_{\mathcal{I}\in\mathcal{G}^p_k} \sum_{I\in\mathcal{I}}\tr(Z_I) \quad \text{s.t.} \quad Z_I\in\mathcal{A}_{I,\succeq}\}
\end{align}
where  $\mathcal{A}_{I,\succeq}:=\{Z_I\succeq 0 ,\supp(Z_I)=I\times I\}$.
\end{mydef}

Note that we can have $\Omega_{k,\succeq}(Z)=+\infty$ even if $Z$ is p.s.d.

\begin{lemm} The polar gauge of $\Omega_{k,\succeq}$ of a smatrix $Y\RR^{p\times p}$ is
\begin{align}
\Omega_{k,\succeq}^{\circ}(Y):= \max\big(\max_{I\in\mathcal{G}^p_k}\lambda_{max}(Y_{II}),0\big)
\end{align}
\end{lemm}

\subsection{Convex formulation}
Let $(x_1,..,x_n)$ be $n$ samples of dimension $p$ and   $\hat{\Sigma}$ the empirical covariance. A natural way to approximate a given sample covariance matrix by a model which the concentration matrix decomposes into a sparse and $k$-low-rank matrix is a regularized using $\ell_1$ norm for recovering sparse component and the convex relaxation of $k$-rank for recovering the $k$-low-rank component. Hence we consider the following convex optimization problem
\begin{align}
\label{opt}
\min_{S,L} f(S-L)+\mu\|S\|_{1}+\lambda\Omega_k(L) \quad s.t. \quad S-L \succeq 0 \quad L \succeq 0,
\end{align}
where  $f$ represents the loss function. Typically,in Graphical model seletion $f$ is the log-likelihood, as proposed in \citet{chandrasekaran2010}. However \textit{logdet} is not Lipschitz preventing us from applying efficient algorithms of the family of conditional gradient, for which there is not convergence results for this setting.  Two other natural losses, which have the advantage of being quadratic, are the second order taylor expansion of the log-likelihood $f_{T}$ and the score matching loss $f_{SM}$, introduced by \citet{hyvarinen2005estimation} and used for graphical model estimation in \citet{lin2016estimation}.
\begin{align}
f_{SM}(K)&:=\frac{1}{2}\tr(K^2 \hat{\Sigma})-\tr(K) \\
f_{T}(K)&:=\frac{1}{2}\|\hat{\Sigma}^{1/2}K\hat{\Sigma}^{1/2}-I\|_2^2.
\end{align}

Since the norm $\Omega_k$ only provides symmetric p.s.d matrices (as a sum of p.s.d rank-one matrices), the nonnegative constraint on $L$ is useless in problem \ref{opt}. \TODO{What about $S-L \succeq 0$}
We choose to consider quadratic losses in this paper, this allowing us to use an efficient algorithm for optimization.

Problem \ref{opt} is a minimization of a quadratic funtion $f$ regularized by an atomic norm $\gamma$  defined as the infimal convolution of $\mu\|.\|_{1}$ and $\lambda\Omega_k(.)$, $\gamma(M)=\inf\{\mu\|A\|_{1}+\lambda\Omega_k(B)|M=A+B\}$
\begin{align}
\label{opt_quad}
\min_{K} f(K)+ \gamma(K)
\end{align}


\subsection{Algorithm}
We use algorithm for optimizing problems regularized by atomic norms from \citet{vinyes2017}
LMO is the polar gauge,  we use truncated power iteration of \citet{yuan2013truncated}, after adding the frobenius norm of the matrix.\\

explain three loops, first we add support, then atoms on support, then as 





